\documentclass{article}
 \usepackage[utf8]{inputenc}
 \usepackage{latexsym}
 \usepackage{amsfonts}
 \begin{document}
 \newcommand{\equalH}[2]{#1 = #2}
\newcommand{\typingH}[2]{#1 : #2}
\newcommand{\lambdaH}[3]{\lambda_{#1 : #2} #3}
\newcommand{\arrowH}[2]{#1 \rightarrow #2}
\newcommand{\equivalenceH}[2]{#1 \simeq #2}


\newcommand{\comprehensionH}[3]{\{ #1 : #2 \mid #3 \}}
\newcommand{\idMapH}[1]{1_{ #1 }}
\newcommand{\fiberH}[2]{\{ #1 \}_{ #2 }}
\newcommand{\appH}[2]{#1 #2}
\newcommand{\defineH}[2]{#1 := #2}
\newcommand{\pairH}[2]{( #1 , #2 )}
\newcommand{\reflexivityH}[2]{r_{ #1 } #2}
\newcommand{\barH}[1]{\bar{ #1 }}
\newcommand{\idPropH}[2]{( #1 = #2 )}
\newcommand{\equivalenceMapH}[2]{E_ { #1 , #2 }}

 
 \textbf{Définition}:
 Un type $A$ est contractible, s'il existe un $a : A$, nommé la centre de contraction, tel que pour tous les $x : A$, $\equalH {a}{x}$.
 
 \textbf{Définition}:
 Une application $f : \arrowH {A}{B}$ est une équivalence, si pour tous les $y : B$, son fibre, $\comprehensionH {x}{A}{\equalH {\appH {f}{x}}{y}}$, est contractible.
 Nous écrivons $\equivalenceH {A}{B}$, s'il existe une équivalence $\arrowH {A}{B}$.
 
 \textbf{Lemme}:
 Pour tout type $A$, l'identité, $\defineH {\idMapH {A}}{\typingH {\lambdaH {x}{A}{x}}{\arrowH {A}{A}}}$, est une équivalence.
 
 \textbf{Démonstration}:
 Pour tout $y : A$, soit $\defineH {\fiberH {y}{A}}{\comprehensionH {x}{A}{\equalH {x}{y}}}$ son fibre par rapport de $\idMapH {A}$ et soit $\defineH {\barH {y}}{\typingH {\pairH {y}{\reflexivityH {A}{y}}}{\fiberH {y}{A}}}$.
 Comme pour tous les $y : A$, $\equalH {\pairH {y}{\reflexivityH {A}{y}}}{y}$, nous pouvons appliquer Id-induction sur $y$, $\typingH {x}{A}$ et $\typingH {z}{\idPropH {x}{y}}$ pour obtenir que \[\equalH {\pairH {x}{z}}{y}\].
 Donc, pour les $y : A$, nous pouvons appliquer $\Sigma$ -élimination sur $\typingH {u}{\fiberH {y}{A}}$ pour obtenir que $\equalH {u}{y}$, de façon que $\fiberH {y}{A}$ soit contractible.
 Alors, $\typingH {\idMapH {A}}{\arrowH {A}{A}}$ est une équivalence.
  $\Box$ 
 
 \textbf{Corollaire}:
 Si $U$ est un univers, alors, pour les $X , Y : U$, \[(*)\arrowH {\equalH {X}{Y}}{\equivalenceH {X}{Y}}\].
 
 \textbf{Démonstration}:
 Nous pouvons appliquer le lemme pour obtenir que pour les $X : U$, $\equivalenceH {X}{X}$.
 Donc, nous pouvons appliquer Id-induction sur $\typingH {X , Y}{U}$ pour obtenir que $(*)$.
  $\Box$ 
 
 
 \textbf{Définition}:
 Un univers $U$ est univalent, si pour les $X , Y : U$, l'application $\equivalenceMapH {X}{Y}: \arrowH {\equalH {X}{Y}}{\equivalenceH {X}{Y}}$ dans $(*)$ est une équivalence.
 
 \end{document}
